\section{Composite systems}
It is actually quite rare that we can label the system with a single quantum number. Any atom will involve spin, position, angular momentum. Other examples might just involve two spin which we observe. So the question is then on how we label those systems. We then have two questions to answer:
\begin{enumerate}
\item How many labels do we need for a system to fully determine its quantum state ?
\item Once I know all the labels, how do I construct the full state out of them ?
\end{enumerate}
We will actually discuss the second question first as it sets the notation for the first question.



\subsection{Entangled States}
 
In AMO we typically would like to characterize is the state of an electron in a hydrogen atom. We need to define its angular momentum label $L$, which might be 0, 1, 2 and also its electron spin $S$, which might be $\{\uparrow, \downarrow\}$. It state is then typically labelled as something like $\ket{L=0, S=\uparrow} = \ket{0,\uparrow}$ etc.

Another, simple example is that of two spins, each one having two possible states $\{\uparrow, \downarrow\}$. This is a standard problem in optical communication, where you send correlated photons with a certain polarization to different people. We will typically call them Alice and Bob \footnote{And if someone wants to listen the person is called Eve}.

We now  would like to understand than if we can disentangle the information about the different labels. Naively, we can now  associate with Alice one set of outcomes and describe it by some state $\ket{\psi_{A}}$ and the Bob has another set $\ket{\psi_{B}}$:
\begin{align}\label{Eq:Local}
	\ket{\psi_A}&= a_{\uparrow} \ket{\uparrow_A}+ a_{\downarrow} \ket{\downarrow_A}\\
	\ket{\psi_B}&= b_{\uparrow} \ket{\uparrow_B}+ b_{\downarrow} \ket{\downarrow_B}
\end{align}

The full state will then be described by the possible outcomes $\{\uparrow_A\uparrow_B,\downarrow_A\uparrow_B,\uparrow_A\downarrow_B, \downarrow_A\downarrow_B\}$. We can then write:
\begin{align}
\ket{\psi} &= \alpha_{\uparrow\uparrow}(\ket{\uparrow_A}\otimes\ket{\uparrow_B})+\alpha_{\uparrow\downarrow}(\ket{\uparrow_A}\otimes\ket{\downarrow_B})+\alpha_{\downarrow\uparrow}(\ket{\downarrow_A}\otimes\ket{\uparrow_B})+\alpha_{\downarrow\downarrow}(\ket{\downarrow_A}\otimes\ket{\downarrow_B})\\
&= \alpha_{\uparrow\uparrow}\ket{\uparrow\uparrow}+\alpha_{\uparrow\downarrow}\ket{\uparrow\downarrow}+\alpha_{\downarrow\uparrow}\ket{\downarrow \uparrow}+\alpha_{\downarrow\downarrow}\ket{\downarrow\downarrow}
\end{align}
So we will typically just plug the labels into a single ket and drop indices, to avoid rewriting the tensor symbol each time. We say that a state is \textit{separable}, if we can write it as a product of the two individual states \eqref{Eq:Local}:
\begin{align}
\ket{\psi} &= \ket{\psi_A}\otimes\ket{\psi_B}\\
&=a_{\uparrow} b_\uparrow \ket{\uparrow\uparrow}+a_{\downarrow} b_\uparrow \ket{\downarrow\uparrow}+a_{\uparrow} b_\downarrow \ket{\uparrow\downarrow}+a_{\downarrow} b_\downarrow \ket{\downarrow\downarrow}
\end{align}

All other states are called \textit{entangled}. The most famous entangled states are the Bell states:
\begin{align}
\ket{\psi_\textrm{Bell}}=\frac{\ket{\uparrow\uparrow}+\ket{\downarrow\downarrow}}{}
\end{align}

As for the shared coin, we


In general we will say that the quantum system is formed by two subsystems $S_1$ and $S_2$. If they are independent we can write each of them as:

\begin{align}
				\ket{\psi_1}&=\sum_m^M a_m \ket{\alpha_m},\\
				\ket{\psi_2}&=\sum_n^N b_n \ket{\beta_n}.
			
\end{align}
In general we will then write:
\begin{align}
\ket{\psi}=\sum_m^M \sum_n^N c_{mn}(\ket{\alpha_m}\otimes \ket{\beta_n}).
\end{align}
So we can determine such a state by $M \times N$ numbers $c_{mn}$ here.  If the states are \textit{separable}, we can write $\ket{\psi}$ as a product of the individual states:
\begin{align}
 \label{eq:psientangled} 
	\ket{\psi}	&=\ket{\psi_1}\otimes\ket{\psi_2}=\left(\sum_m^M a_m \ket{\alpha_m}\right) \otimes \left(\sum_n^N b_n \ket{\beta_n}\right)
\end{align}
\begin{align}
\ket{\psi}	&=\sum_m^M \sum_n^N a_m b_n \ket{\alpha_m} \otimes \ket{\beta_n}. \label{eq:psientangled3} 
\end{align}

Separable states thus only describes a small subset of all possible states. 