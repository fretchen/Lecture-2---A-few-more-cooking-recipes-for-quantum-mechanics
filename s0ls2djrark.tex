\section{Motion of a Point-Like Particle}
		\subsection{Wave Functions}
			In the following, we consider a point-like particle and its observables $\vv{r}$ and $\vv{p}$. The possible basis states can be constructed from the observables we measure. We thus get $\ket{\vv{r}}$ and $\ket{\vv{p}}$ as basis states. Since the particle is localized at one point, we get
			\begin{align}
				\braket{\vv{r}|\vv{r}'} =\delta(\vv{r}-\vv{r}').
			\end{align}
			%
			With the completeness relation, we can express the wave function $\ket{\psi}$ of the particle as a linear combination of all possible position states $\ket{\vv{r}}$:
			\begin{align}
				\ket{\psi}=\int \dif \vv{r} \ket{\vv{r}}\braket{\vv{r}|\psi},
			\end{align}
			where $\psi(\vv{r})=\braket{\vv{r}|\psi}$ and $\psi^*\psi$ is the probability density.
			%
			The eigenstates of $\hat{\vv{p}}$ are
			\begin{align}
				\ket{\vv{p}}&=\int \dif \vv{r} \ket{\vv{r}} \braket{\vv{r}|\vv{p}} \\
				&= \frac{1}{(2\pi\hbar)^{\frac{3}{2}}} \int \dif \vv{r} \exp{{i \vv{p}\cdot\vv{r}}/{\hbar}}\ket{\vv{r}},
			\end{align}
			where the exponential function in the integral can be interpreted as a plane wave.
			%
			If we expand $\ket{\psi}$ in terms of $\ket{p}$, we get
			\begin{align}
				\ket{\psi}=\int \dif \vv{p} \ket{\vv{p}}\underbrace{\braket{\vv{p}|\psi}}_{=\phi(\vv{p})}.
			\end{align}
			The second factor in the integral can be expanded as follows:
			\begin{align}
				\phi(\vv{p})	&=\braket{\vv{p}|\psi}\\
				&=\int\dif\vv{r}\braket{\vv{p}|\vv{r}}\braket{\vv{r}|\psi}\\
				&=\frac{1}{(2\pi\hbar)^{\frac{3}{2}}} \int\dif\vv{r}\exp{-i\vv{p}\cdot\vv{r}/{\hbar}}\psi(\vv{r}) \label{eq:fourier}
			\end{align}
			Equation \eqref{eq:fourier} shows that $\phi(\vv{p})$ and $\psi(\vv{r})$ are Fourier transforms of each other.

\section{Translations of $\ket{\psi}$ in Time, Space and Momentum}

\subsection{Time: Schrödinger Equation}
	In the following, we would like to get a basic intuition of how quantum mechanics works. For the Schrödinger equation
	\begin{align}
	i\hbar \partial_t\ket{\psi(t)} = \hat{H}\ket{\psi(t)} \label{eq:schrodinger2}
	\end{align}
	we get the formal solution
			\begin{align}
				\ket{\psi(t)}=\hat{U}(t,0)\ket{\psi(0)}, \label{eq:psit}
			\end{align}
			where
			\begin{align}
				\hat{U}(t,0) = \exp{-i{\hat{H}t}/{\hbar}} \label{eq:timeevolutionoperator}
			\end{align}
	is the time evolution operator. If we write out the Hamiltonian of the point-like particle in \eqref{eq:schrodinger2}, we get
	\begin{align}
		i\hbar \partial_t\psi(\vec{r},t) = -\frac{\hbar^2}{2m}\nabla\psi(\vv{r},t)+V(\vv{r},t)\psi(\vv{r},t).
	\end{align}
The wave function $\psi(\vec{r},t)$ at time $t$ can be determined with the help of \eqref{eq:psit} and \eqref{eq:timeevolutionoperator}:
			\begin{align}
				\psi(\vv{r},t)	&=\braket{\vv{r}|\psi(t)}=\braket{\vv{r}|\hat{U}(t,0)|\psi(0)}\\
				&=\int\dif\vv{r}'\bra{\vv{r}}\hat{U}(t,0)\ket{\vv{r}'}\underbrace{\bra{\vv{r}'}\psi(0)\rangle}_{=\psi(\vv{r'}, 0)}.
			\end{align}
The first factor in the integral describes the propagation of a wave packet localized at $\vv{r}'$ during the time $t$. The second factor is the wave function of a particle localized at $\vv{r}'$ at a time $t=0$. %Is wave function of particle?

\subsection{Space and Momentum}

Just like $\hat{H}$ is a generator of translation in time, $\hat{\vv{p}}$ is the generator for translation in space and $\hat{\vv{r}}$ is the generator for translation in momentum:
				\begin{align}
					\hat{\mathcal{T}}_R(\Delta{\vv{r}}) \ket{\vec{r}} &= \exp{-i\hat{\vv{p}}\cdot{\Delta\vv{r}}/{\hbar}}\ket{\vv{r}}=\ket{\vv{r}+\Delta\vv{r}},\\
					\hat{\mathcal{T}}_R(\Delta{\vv{p}}) \ket{\vec{p}} &= \exp{-i\hat{\vv{r}}\cdot{\Delta\vv{p}}/{\hbar}}\ket{\vv{p}}=\ket{\vv{p}+\Delta\vv{p}}.
				\end{align}