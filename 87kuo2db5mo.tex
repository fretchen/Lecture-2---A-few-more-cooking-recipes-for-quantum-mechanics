\section{Statistical Mixtures and Density Operator}

If only the subsystem of a pure quantum state is accessible to measurements, or the state of the system is not known at the microscopic level (statistical ensemble!), the state of the system has to be described by a Hermitian density operator

\begin{align}
 \hat{\rho} = \sum_{n=1}^N p_n \ket{\phi_n}\bra{\phi_n}.
\end{align}
Here, $\bra{\phi_n}$ are the eigenstates of $\hat{\rho}$, and $p_n$ are the probabilities to find the system in the respective states $\ket{\phi_n}$. For a pure state $\ket{\psi}$, we get $p_n=1$ for only one value of $n$. For every other $n$, the probabilities vanish. We thus obtain a ``pure'' density operator $\rhohat_{\text{pure}}$ which has the properties of a projection operator:

\begin{align}
					\rhohat_{\text{pure}} = \ket{\psi}\bra{\psi} \qquad \Longleftrightarrow \qquad \rhohat^2 = \rhohat.
\end{align}
				
Let us now have a brief look at the properties of the density operator:
\begin{itemize}
    \item The trace of the density operator is the sum of all probabilities $p_n$:
	%
	\begin{align}
	    \tr{\rhohat} = \sum p_n = 1.
	\end{align}
				%
	\item 
\end{itemize}
			%
With this knowledge we can now determine the result of a measurement of an observable $A$ belonging to an operator $\hat{A}$. For the pure state $\ket{\psi}$ we get:
%
\begin{align}
				\langle \hat{A}\rangle = \bra{\psi} \hat{A} \ket{\psi}.
\end{align}
			%
For a mixed state we get:
			%
\begin{align}
	\langle \hat{A}\rangle = \tr{\rhohat \cdot \hat{A}} = \sum_n {p_n} \bra{\phi_n} \hat{A} \ket{\phi_n}.
\end{align}
The time evolution of the density operator can be expressed with the von Neumann equation:
\begin{align}
	i\hbar \partial_{t}\rhohat(t) = [\hat{H}(t),\rhohat(t)].
\end{align}
\textbf{Note.} 
If we consider a system $S = S_1 \otimes S_2$ comprised of two subsystems $S_1$, $S_2$, then the density operator $\hat{\rho}_i$ of  subsystem $i$ is
					%
					\begin{align}
						\rhohat_1=&\trarb{2}{\rhohat},\\
						\rhohat_2=&\trarb{1}{\rhohat},
					\end{align}
					%
					where $\hat{\rho}=\ket{\psi}\bra{\psi}$ and $\trarb{j}{\rhohat}$ is the trace over the Hilbert space of subsystem $j$.
					
\subsection{Complete Set of Commuting Observables}

A set of commuting operators $\{\hat{A},\hat{B},\hat{C},\cdots,\hat{X}\}$ is considered a complete set if their common eigenbasis is unique. Thus, the measurement of all quantities $\{A,B,\cdots,X\}$ will determine the system uniquely. The clean identification of such a Hilbert space can be quite challenging and a nice way of its measurment even more. Coming back to our previous examples:

\begin{enumerate}
\item Performing the full spectroscopy of the atom. Even for the hydrogen atom we will see that the full answer can be rather involved...
\item The occupation number is rather straight forward. However, we have to be careful that we really collect a substantial amount of the photons etc.
\item Are we able to measure the full position information ? What is the resolution of the detector and the point-spread function ?
\item Here it is again rather clean to put a very efficient detector at the output of the two arms ...
\item What are the components of the spin that we can access ? The $z$ component does not commute with the other components, so what should we measure ?
\end{enumerate}