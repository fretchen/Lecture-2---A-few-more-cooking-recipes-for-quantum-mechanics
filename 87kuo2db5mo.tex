\section{Statistical Mixtures and Density Operator}

Having set up the formalism for writing down the full quantum state with plenty of labels, we have to solve the next problem. As an experimentalist, you will rarely measure all of them. This means that you only perform a partial measurement and you have only partial information of the system. The extreme case is the thermodynamic ensemble, where we measure only temperature to describe $10^{23}$ particles.

A similiar problem arises for Alice and Bob. They typically measure the state of the qubit in their lab without knowing what the other did. So they need some way to describe the system locally. This is done through the density operator approach.

In the density operator approach the state of the system is described by a Hermitian density operator

\begin{align}
 \hat{\rho} = \sum_{n=1}^N p_n \ket{\phi_n}\bra{\phi_n}.
\end{align}
Here, $\bra{\phi_n}$ are the eigenstates of $\hat{\rho}$, and $p_n$ are the probabilities to find the system in the respective states $\ket{\phi_n}$. The trace of the density operator is the sum of all probabilities $p_n$:
\begin{align}
  \tr{\rhohat} = \sum p_n = 1.
\end{align}

For a pure state $\ket{\psi}$, we get $p_n=1$ for only one value of $n$. For every other $n$, the probabilities vanish. We thus obtain a ``pure'' density operator $\rhohat_{\text{pure}}$ which has the properties of a projection operator:

\begin{align}
	\rhohat_{\text{pure}} = \ket{\psi}\bra{\psi} \qquad \Longleftrightarrow \qquad \rhohat^2 = \rhohat.
\end{align}
For the simple qubit we then have:
\begin{align}
  \rhohat &= \left(\alpha_\uparrow\ket{\uparrow}+\alpha_\downarrow\ket{\downarrow}\right)\left(\alpha_\uparrow\bra{\uparrow}+\alpha_\downarrow\bra{\downarrow}\right)\\
  &= \left(\alpha_\uparrow \alpha_\uparrow\ket{\uparrow}\bra{\uparrow}+\alpha_\downarrow^*\ket{\downarrow}\right)\left(\alpha_\uparrow\bra{\uparrow}+\alpha_\downarrow\bra{\downarrow}\right)
\end{align}

For a thermal state on the other hand we have:
\begin{align}
	\rhohat_{\text{thermal}} = \sum_{n=1}^N \frac{e^{-\frac{E_n}{k_BT}}}{Z} \ket{\phi_n}\bra{\phi_n}\text{ with }Z = \sum_{n=1}^N e^{-\frac{E_n}{k_BT}}
\end{align}
With this knowledge we can now determine the result of a measurement of an observable $A$ belonging to an operator $\hat{A}$. For the pure state $\ket{\psi}$ we get:
%
\begin{align}
				\langle \hat{A}\rangle = \bra{\psi} \hat{A} \ket{\psi}.
\end{align}
			%
For a mixed state we get:
			%
\begin{align}
	\langle \hat{A}\rangle = \tr{\rhohat \cdot \hat{A}} = \sum_n {p_n} \bra{\phi_n} \hat{A} \ket{\phi_n}.
\end{align}
The time evolution of the density operator can be expressed with the von Neumann equation:
\begin{align}
	i\hbar \partial_{t}\rhohat(t) = [\hat{H}(t),\rhohat(t)].
\end{align}

\subsection{Back to partial measurements}

We can now come back to the correlated photons sent to Alice and Bob, sharing a Bell pair. Let us write the system as $S = S_A \otimes S_B$.  comprised of two subsystems $S_1$, $S_2$, then the density operator $\hat{\rho}_i$ of  subsystem $i$ is
					%
					\begin{align}
						\rhohat_1=&\trarb{2}{\rhohat},\\
						\rhohat_2=&\trarb{1}{\rhohat},
					\end{align}
					%
					where $\hat{\rho}=\ket{\psi}\bra{\psi}$ and $\trarb{j}{\rhohat}$ is the trace over the Hilbert space of subsystem $j$.
