\subsection{Statistical Mixtures and Density Operator}

If only the subsystem of a pure quantum state is accessible to measurements, or the state of the system is not known at the microscopic level (statistical ensemble!), the state of the system has to be described by a Hermitian density operator

\begin{align}
 \hat{\rho} = \sum_{n=1}^N p_n \ket{\phi_n}\bra{\phi_n}.
\end{align}
Here, $\bra{\phi_n}$ are the eigenstates of $\hat{\rho}$, and $p_n$ are the probabilities to find the system in the respective states $\ket{\phi_n}$.

			Let us now have a brief look at the properties of the density operator:
			\begin{itemize}
				\item The trace of the density operator is the sum of all probabilities $p_n$:
				%
				\begin{align}
					\tr{\rhohat} = \sum p_n = 1.
				\end{align}
				%
				\item For a pure state $\ket{\psi}$, we get $p_n=1$ for only one value of $n$. For every other $n$, the probabilities vanish. We thus obtain a ``pure'' density operator $\rhohat_{\text{pure}}$ which has the properties of a projection operator:
				%
				\begin{align}
					\rhohat_{\text{pure}} = \ket{\psi}\bra{\psi} \qquad \Longleftrightarrow \qquad \rhohat^2 = \rhohat.
				\end{align}
			\end{itemize}
			%
			With this knowledge we can now determine the result of a measurement of an observable $A$ belonging to an operator $\hat{A}$. For the pure state $\ket{\psi}$ we get:
%
			\begin{align}
				\bra{\psi} \hat{A} \ket{\psi}.
			\end{align}
			%
			For a mixed state we get:
			%
			\begin{align}
				\tr{\rhohat \cdot \hat{A}} = \sum_n {p_n} \bra{\phi_n} \hat{A} \ket{\phi_n}.
			\end{align}
			The time evolution of the density operator can be expressed with the von Neumann equation:
			\begin{align}
				i\hbar \partial_{t}\rhohat(t) = [\hat{H}(t),\rhohat(t)].
			\end{align}
				\paragraph{Example.} 
					If we consider a system $S = S_1 \otimes S_2$ comprised of two subsystems $S_1$, $S_2$, then the density operator $\hat{\rho}_i$ of  subsystem $i$ is
					%
					\begin{align}
						\rhohat_1=&\trarb{2}{\rhohat},\\
						\rhohat_2=&\trarb{1}{\rhohat},
					\end{align}
					%
					where $\hat{\rho}=\ket{\psi}\bra{\psi}$ and $\trarb{j}{\rhohat}$ is the trace over the Hilbert space of subsystem $j$ (cf. first tutorial).
					%\\
					%\rhohat_2=\trarb{2}{\rhohat}
		\subsection{Important Consequences of the Principles}

			\subsubsection{Uncertainty Relation}
				The product of the variances of two noncommuting operators has a lower limit:
				\begin{align}
					\Delta \hat{A} \cdot \Delta \hat{B} \geq \frac{1}{2} \left| \Braket{\left[\hat{A},\hat{B}\right]} \right|,
				\end{align}
				where the variance is defined as $\Delta \hat{A} = \sqrt{\Braket{\hat{A}^2}-\Braket{\hat{A}}^2}$.

				\paragraph{Examples.}
					\begin{align}
						\left[ \hat{x}, \hat{p} \right] &= i \hbar \\
						\left[ \hat{J}_i , \hat{J}_j \right] &= i \hbar \epsilon_{ijk} \hat{J}_k
					\end{align}
				\paragraph{Note.} This is a statement about the \emph{state} itself, and not the measurement!

			\subsubsection{Ehrenfest Theorem}
				With the Ehrenfest theorem, one can determine the time evolution of the expectation value of an operator $\hat{A}$:
				\begin{align}
					\diff{}{t}\Braket{\hat{A}}=\frac{1}{i\hbar}\Braket{\left[\hat{A},\hat{H}\right]}+\Braket{\diffp{\hat{A}}{t}}. \label{eq:ehrenfest}
				\end{align}
				%
				If $\hat{A}$ is time-independent and $\left[\hat{A},\hat{H}\right]=0$, the expectation value $\Braket{\hat{A}}$ is a constant of the motion.

	\section{Motion of a Point-Like Particle}
		\subsection{Wave Functions}
			In the following, we consider a point-like particle and its observables $\vv{r}$ and $\vv{p}$. The possible basis states can be constructed from the observables we measure. We thus get $\ket{\vv{r}}$ and $\ket{\vv{p}}$ as basis states. Since the particle is localized at one point, we get
			\begin{align}
				\braket{\vv{r}|\vv{r}'} =\delta(\vv{r}-\vv{r}').
			\end{align}
			%
			With the completeness relation, we can express the wave function $\ket{\psi}$ of the particle as a linear combination of all possible position states $\ket{\vv{r}}$:
			\begin{align}
				\ket{\psi}=\int \dif \vv{r} \ket{\vv{r}}\braket{\vv{r}|\psi},
			\end{align}
			where $\psi(\vv{r})=\braket{\vv{r}|\psi}$ and $\psi^*\psi$ is the probability density.
			%
			The eigenstates of $\hat{\vv{p}}$ are
			\begin{align}
				\ket{\vv{p}}&=\int \dif \vv{r} \ket{\vv{r}} \braket{\vv{r}|\vv{p}} \\
				&= \frac{1}{(2\pi\hbar)^{\frac{3}{2}}} \int \dif \vv{r} \eexp{{i \vv{p}\cdot\vv{r}}/{\hbar}}\ket{\vv{r}},
			\end{align}
			where the exponential function in the integral can be interpreted as a plane wave.
			%
			If we expand $\ket{\psi}$ in terms of $\ket{p}$, we get
			\begin{align}
				\ket{\psi}=\int \dif \vv{p} \ket{\vv{p}}\underbrace{\braket{\vv{p}|\psi}}_{=\phi(\vv{p})}.
			\end{align}
			The second factor in the integral can be expanded as follows:
			\begin{align}
				\phi(\vv{p})	&=\braket{\vv{p}|\psi}\\
				&=\int\dif\vv{r}\braket{\vv{p}|\vv{r}}\braket{\vv{r}|\psi}\\
				&=\frac{1}{(2\pi\hbar)^{\frac{3}{2}}} \int\dif\vv{r}\eexp{-i\vv{p}\cdot\vv{r}/{\hbar}}\psi(\vv{r}) \label{eq:fourier}
			\end{align}
			Equation \eqref{eq:fourier} shows that $\phi(\vv{p})$ and $\psi(\vv{r})$ are Fourier transforms of each other.

	\subsection{Translations of $\ket{\psi}$ in Time, Space and Momentum}

		\subsubsection{Time: Schrödinger Equation}
			In the following, we would like to get a basic intuition of how quantum mechanics works. For the Schrödinger equation
			\begin{align}
				i\hbar \diffp{\ket{\psi(t)}}{t} = \hat{H}\ket{\psi(t)} \label{eq:schrodinger2}
			\end{align}
			we get the formal solution
			\begin{align}
				\ket{\psi(t)}=\hat{U}(t,0)\ket{\psi(0)}, \label{eq:psit}
			\end{align}
			where
			\begin{align}
				\hat{U}(t,0) = \eexp{-i{\hat{H}t}/{\hbar}} \label{eq:timeevolutionoperator}
			\end{align}
			is the time evolution operator. If we write out the Hamiltonian of the point-like particle in \eqref{eq:schrodinger2}, we get
			\begin{align}
				i\hbar \diffp{\psi(\vec{r},t)}{t} = -\frac{\hbar^2}{2m}\Laplace\psi(\vv{r},t)+V(\vv{r},t)\psi(\vv{r},t).
			\end{align}
			The wave function $\psi(\vec{r},t)$ at time $t$ can be determined with the help of \eqref{eq:psit} and \eqref{eq:timeevolutionoperator}:
			\begin{align}
				\psi(\vv{r},t)	&=\braket{\vv{r}|\psi(t)}=\braket{\vv{r}|\hat{U}(t,0)|\psi(0)}\\
				&=\int\dif\vv{r}'\bra{\vv{r}}\hat{U}(t,0)\overunderbraces{&\br{2}{\mathbb{1}}}{&\ket{\vv{r}'}&\bra{\vv{r}'}&\psi(0)\rangle}{&&\br{2}{\psi(\vv{r}',0)}}.
			\end{align}
			% 
			%\overunderbraces{&\br{2}{a}}{&&}{&&\br{2}{}}
			The first factor in the integral describes the propagation of a wave packet localized at $\vv{r}'$ during the time $t$. The second factor is the wave function of a particle localized at $\vv{r}'$ at a time $t=0$. %Is wave function of particle?

			\subsubsection{Space and Momentum}

				Just like $\hat{H}$ is a generator of translation in time, $\hat{\vv{p}}$ is the generator for translation in space and $\hat{\vv{r}}$ is the generator for translation in momentum:
				\begin{align}
					\hat{\mathcal{T}}_R(\Delta{\vv{r}}) \ket{\vec{r}} &= \eexp{-i\hat{\vv{p}}\cdot{\Delta\vv{r}}/{\hbar}}\ket{\vv{r}}=\ket{\vv{r}+\Delta\vv{r}},\\
					\hat{\mathcal{T}}_R(\Delta{\vv{p}}) \ket{\vec{p}} &= \eexp{-i\hat{\vv{r}}\cdot{\Delta\vv{p}}/{\hbar}}\ket{\vv{p}}=\ket{\vv{p}+\Delta\vv{p}}.
				\end{align}