\section{The Heisenberg picture}
As mentionned in the first lecture it can benefitial to work in the Heisenberg picture instead of the Schrödinger picture. This approach is widely used in the field of many-body physics, as it underlies the formalism of the second quantization. To make the connection with  the Schrödinger picture we should remember that we have the formal solution of
\begin{align}
\ket{\psi(t)} = \eexp{-i\hat{H}t}\ket{\psi(t)}
\end{align}
\begin{align}
    \hat{A}_H=\eexp{i{\hat{H} t}/{\hbar}} \hat{A}_S \eexp{-i{\hat{H} t}/{\hbar}}
\end{align}
					where $\eexp{-i{\hat{H} t}/{\hbar}}$ is a time evolution operator (N.B.: $\hat{H}_S = \hat{H}_H$). The time evolution of $\hat{A}_H$ is:
					\begin{align}
						\notag \diff{}{t} \hat{A}_H &=&& \frac{i}{\hbar}\hat{H}\eexp{i{\hat{H}t}/{\hbar}}\hat{A}_S \eexp{-i{\hat{H} t}/{\hbar}}\\ 
						&&-&\frac{i}{\hbar} \eexp{i{\hat{H} t}/{\hbar}}\hat{A}_S \eexp{-i{\hat{H}t}/{\hbar}}\hat{H}+\diffp{\hat{A}_H}{t}\\
						&=&& \frac{i}{\hbar}\left[\hat{H},\hat{A}_H\right] + \eexp{i{\hat{H}t}/{\hbar}}\diffp{\hat{A}_S}{t}\eexp{-i{\hat{H}t}/{\hbar}}
					\end{align}

					In the Heisenberg picture the state vectors are time-in\-de\-pen\-dent:
					\begin{align}
						\ket{\psi}_H \equiv \ket{\psi(t=0)}=\eexp{i{\hat{H}}t/{\hbar}} \ket{\psi(t)}.
					\end{align}
					Therefore, the results of measurements are the same in both pictures:
					\begin{align}
						\bra{\psi(t)}\hat{A}\ket{\psi(t)} = \bra{\psi}_H \hat{A}_H \ket{\psi}_H.
					\end{align}
					For example, applying this to the spin operators yields:
					\begin{align}
						\diff{}{t}\hat{s}_{i,H}=\frac{i}{\hbar}\left[\hat{H},\hat{s}_{i,H}\right].
					\end{align}
