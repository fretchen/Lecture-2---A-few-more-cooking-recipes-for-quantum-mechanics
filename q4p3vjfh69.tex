\section{Important Consequences of the Principles}
\subsection{Uncertainty Relation}

The product of the variances of two noncommuting operators has a lower limit:
\begin{align}
    \Delta \hat{A} \cdot \Delta \hat{B} \geq \frac{1}{2} \left| \braket{\left[\hat{A},\hat{B}\right]} \right|,
\end{align}
where the variance is defined as $\Delta \hat{A} = \sqrt{\braket{\hat{A}^2}-\braket{\hat{A}}^2}$.

				\paragraph{Examples.}
					\begin{align}
						\left[ \hat{x}, \hat{p} \right] &= i \hbar \\
						\left[ \hat{J}_i , \hat{J}_j \right] &= i \hbar \epsilon_{ijk} \hat{J}_k
					\end{align}
				\paragraph{Note.} This is a statement about the \emph{state} itself, and not the measurement!

			\subsubsection{Ehrenfest Theorem}
				With the Ehrenfest theorem, one can determine the time evolution of the expectation value of an operator $\hat{A}$:
				\begin{align}
					\diff{}{t}\Braket{\hat{A}}=\frac{1}{i\hbar}\Braket{\left[\hat{A},\hat{H}\right]}+\Braket{\diffp{\hat{A}}{t}}. \label{eq:ehrenfest}
				\end{align}
				%
				If $\hat{A}$ is time-independent and $\left[\hat{A},\hat{H}\right]=0$, the expectation value $\Braket{\hat{A}}$ is a constant of the motion.

	