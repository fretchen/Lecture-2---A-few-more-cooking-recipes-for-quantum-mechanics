\section{Important Consequences of the Principles}
\subsection{Uncertainty Relation}

The product of the variances of two noncommuting operators has a lower limit:
\begin{align}
    \Delta \hat{A} \cdot \Delta \hat{B} \geq \frac{1}{2} \left| \braket{\left[\hat{A},\hat{B}\right]} \right|,
\end{align}
where the variance is defined as $\Delta \hat{A} = \sqrt{\braket{\hat{A}^2}-\braket{\hat{A}}^2}$.

\textbf{Examples.}
\begin{align}
	\left[ \hat{x}, \hat{p} \right] &= i \hbar \\
\left[ \hat{J}_i , \hat{J}_j \right] &= i \hbar \epsilon_{ijk} \hat{J}_k
\end{align}
				
\textbf{Note.} This is a statement about the \emph{state} itself, and not the measurement!

\subsection{Ehrenfest Theorem}
With the Ehrenfest theorem, one can determine the time evolution of the expectation value of an operator $\hat{A}$:
\begin{align}
 \frac{d}{dt}\braket{\hat{A}}=\frac{1}{i\hbar}\braket{\left[\hat{A},\hat{H}\right]}+\braket{\partial_t{\hat{A}}{t}}. \label{eq:ehrenfest}
\end{align}
If $\hat{A}$ is time-independent and $\left[\hat{A},\hat{H}\right]=0$, the expectation value $\braket{\hat{A}}$ is a constant of the motion.

\subsection{Complete Set of Commuting Observables}

A set of commuting operators $\{\hat{A},\hat{B},\hat{C},\cdots,\hat{X}\}$ is considered a complete set if their common eigenbasis is unique. Thus, the measurement of all quantities $\{A,B,\cdots,X\}$ will determine the system uniquely. The clean identification of such a Hilbert space can be quite challenging and a nice way of its measurment even more. Coming back to our previous examples:

\begin{enumerate}
\item Performing the full spectroscopy of the atom. Even for the hydrogen atom we will see that the full answer can be rather involved...
\item The occupation number is rather straight forward. However, we have to be careful that we really collect a substantial amount of the photons etc.
\item Are we able to measure the full position information ? What is the resolution of the detector and the point-spread function ?
\item Here it is again rather clean to put a very efficient detector at the output of the two arms ...
\item What are the components of the spin that we can access ? The $z$ component does not commute with the other components, so what should we measure ?
\end{enumerate}